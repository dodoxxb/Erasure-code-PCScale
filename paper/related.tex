\setlength{\parindent}{0pt}
\section{RELATED WORK}

\textbf{Scaling mechansim.}Recent years, in order to raise the efficiency of 
scaling process, a large amount of scaling mechansim have been proposed for 
erasure-coded distributed storage systems and RAID in the literature, which 
are focus on minimizing the cost of data block  migration and parity block 
updates. However, such mechansims don't achieve the minimum cost of scaling 
procss. In this work, we concentrates on scaling mechanism for erasure-coded
distributed storage, whose performance is more efficient than previous works
no matter scaling for one time or more. \\
\textbf{Parallel execution and compression transmission.}To the best of our 
knowledge, we are the first one to realize that make the two necessary 
operations of scaling process execute parallelly(i.e data blocks migration 
and parity blocks update). PCScale utilize the encoding relationship 
between data blocks and parity block to separate above two operations that 
will reduce the latency time of scaling process. Compared to NCScale, 
PCScale compute delta parity blocks not only make use of data blocks, but 
also corresponding parity blocks, which can achieve the purpose of 
compression transmission and minimize disks I/Os of scaling process.  \\ 
\textbf{Repair-Scaling trade-off for Erasure Codes.}There has been plenty of
studies on adaptive storage scheme base on frequency of data access and 
matched transition algorithm while data with different properties is stored
in different ways. However, such approachs focus on raising recovery 
performance for erasure-coded storage systems. What's more, Wu et al. 
analyzes the optimal repair-scaling trade-off curve. In this work, we 
leverage the relationship between stripe width and repair performance to 
achieve the different repair and scaling demands of hot data and cold data
after multiple scaling process.

\section{CONCLUSION}
